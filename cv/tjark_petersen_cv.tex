%%%%%%%%%%%%%%%%%
% This is an example CV created using altacv.cls (v1.6.3, 04 Oct 2021) written by
% LianTze Lim (liantze@gmail.com), based on the
% Cv created by BusinessInsider at http://www.businessinsider.my/a-sample-resume-for-marissa-mayer-2016-7/?r=US&IR=T
%
%% It may be distributed and/or modified under the
%% conditions of the LaTeX Project Public License, either version 1.3
%% of this license or (at your option) any later version.
%% The latest version of this license is in
%%    http://www.latex-project.org/lppl.txt
%% and version 1.3 or later is part of all distributions of LaTeX
%% version 2003/12/01 or later.
%%%%%%%%%%%%%%%%

%% Use the "normalphoto" option if you want a normal photo instead of cropped to a circle
% \documentclass[10pt,a4paper,normalphoto]{altacv}

\documentclass[10pt,a4paper,ragged2e,withhyper]{altacv}

%% AltaCV uses the fontawesome5 package.
%% See http://texdoc.net/pkg/fontawesome5 for full list of symbols.

% Change the page layout if you need to
\geometry{left=1.25cm,right=1.25cm,top=1.5cm,bottom=1.5cm,columnsep=1.2cm}

% The paracol package lets you typeset columns of text in parallel
\usepackage{paracol}


% Change the font if you want to, depending on whether
% you're using pdflatex or xelatex/lualatex
\ifxetexorluatex
  % If using xelatex or lualatex:
  \setmainfont{Lato}
\else
  % If using pdflatex:
  \usepackage[default]{lato}
\fi

% Change the colours if you want to
\definecolor{CustomDark}{HTML}{00695C}
\definecolor{CustomMedium}{HTML}{00897B}
\definecolor{CustomLight}{HTML}{26A69A}
\definecolor{VividPurple}{HTML}{00695C}
\definecolor{SlateGrey}{HTML}{2E2E2E}
\definecolor{LightGrey}{HTML}{666666}
% \colorlet{name}{black}
\colorlet{tagline}{CustomMedium}
\colorlet{heading}{CustomMedium}
\colorlet{headingrule}{CustomMedium}
% \colorlet{subheading}{PastelRed}
\colorlet{accent}{CustomMedium}
\colorlet{emphasis}{SlateGrey}
\colorlet{body}{LightGrey}

% Change some fonts, if necessary
% \renewcommand{\namefont}{\Huge\rmfamily\bfseries}
% \renewcommand{\personalinfofont}{\footnotesize}
% \renewcommand{\cvsectionfont}{\LARGE\rmfamily\bfseries}
% \renewcommand{\cvsubsectionfont}{\large\bfseries}

% Change the bullets for itemize and rating marker
% for \cvskill if you want to
\renewcommand{\itemmarker}{{\small\textbullet}}
\renewcommand{\ratingmarker}{\faCircle}

\usepackage{tabularx}

\newcommand{\course}[2]{\href{https://kurser.dtu.dk/course/#1}{\textit{#1 - #2}}}
\newcommand{\link}[2]{\href{#1}{\textit{#2}}}
\newcommand{\cvskillLvl}[2]{%
  \noindent \begin{tabularx}{\columnwidth}{@{}p{3.9cm}X}%
  \textcolor{emphasis}{\textbf{#1}} & %
  \textcolor{body}{#2}\end{tabularx}\par%
}



%% Use (and optionally edit if necessary) this .cfg if you
%% want to use an author-year reference style like APA(6)
%% for your publication list
% When using APA6 if you need more author names to be listed
% because you're e.g. the 12th author, add apamaxprtauth=12
\usepackage[backend=biber,style=apa6,sorting=ydnt]{biblatex}
\defbibheading{pubtype}{\cvsubsection{#1}}
\renewcommand{\bibsetup}{\vspace*{-\baselineskip}}
\AtEveryBibitem{\makebox[\bibhang][l]{\itemmarker}}
\setlength{\bibitemsep}{0.25\baselineskip}
\setlength{\bibhang}{1.25em}


%% Use (and optionally edit if necessary) this .cfg if you
%% want an originally numerical reference style like IEEE
%% for your publication list
% \usepackage[backend=biber,style=ieee,sorting=ydnt]{biblatex}
%% For removing numbering entirely when using a numeric style
\setlength{\bibhang}{1.25em}
\DeclareFieldFormat{labelnumberwidth}{\makebox[\bibhang][l]{\itemmarker}}
\setlength{\biblabelsep}{0pt}
\defbibheading{pubtype}{\cvsubsection{#1}}
\renewcommand{\bibsetup}{\vspace*{-\baselineskip}}


%% sample.bib contains your publications
\addbibresource{sample.bib}

\begin{document}
\name{Tjark Petersen}
\tagline{}
% Cropped to square from https://en.wikipedia.org/wiki/Marissa_Mayer#/media/File:Marissa_Mayer_May_2014_(cropped).jpg, CC-BY 2.0
%% You can add multiple photos on the left or right
\photoR{2.5cm}{pass2020_s}
% \photoL{2cm}{Yacht_High,Suitcase_High}
\personalinfo{%
  % Not all of these are required!
  % You can add your own with \printinfo{symbol}{detail}
  
  
  \email{tjark-petersen@gmx.de}
  \phone{+45 60 55 38 40}
   
  \location{Copenhagen, Denmark}
  \mailaddress{Strandvejen 143, 2900 Hellerup}
  
  \homepage{tjarker.github.io}
  \github{tjarker}
  \linkedin{tjarkpetersen}
  \orcid{0000-0002-0239-511X} % Obviously making this up too.
  %% You can add your own arbitrary detail with
  %% \printinfo{symbol}{detail}[optional hyperlink prefix]
  % \printinfo{\faPaw}{Hey ho!}
  %% Or you can declare your own field with
  %% \NewInfoFiled{fieldname}{symbol}[optional hyperlink prefix] and use it:
  % \NewInfoField{gitlab}{\faGitlab}[https://gitlab.com/]
  % \gitlab{your_id}
	%%
  %% For services and platforms like Mastodon where there isn't a
  %% straightforward relation between the user ID/nickname and the hyperlink,
  %% you can use \printinfo directly e.g.
  % \printinfo{\faMastodon}{@username@instace}[https://instance.url/@username]
  %% But if you absolutely want to create new dedicated info fields for
  %% such platforms, then use \NewInfoField* with a star:
  % \NewInfoField*{mastodon}{\faMastodon}
  %% then you can use \mastodon, with TWO arguments where the 2nd argument is
  %% the full hyperlink.
  % \mastodon{@username@instance}{https://instance.url/@username}
}

\makecvheader

%% Depending on your tastes, you may want to make fonts of itemize environments slightly smaller
\AtBeginEnvironment{itemize}{\small}

%% Set the left/right column width ratio to 6:4.
\columnratio{0.6}

% Start a 2-column paracol. Both the left and right columns will automatically
% break across pages if things get too long.
\begin{paracol}{2}

\cvsection{Experience}

\cvevent{Student Project}{Technical University of Denmark}{Sep 2021 -- Dec 2021}{Copenhagen, Denmark}
\begin{itemize}
\item Work on constrained random assembly program generation for processor verification purposes as an addition to the \link{https://github.com/chiselverify/chiselverify}{ChiselVerify} hardware verification framework
\end{itemize}

\divider

\cvevent{Teaching Assistant}{Technical University of Denmark}{Sept 2020 -- Ongoing}{Copenhagen, Denmark}
\begin{itemize}
    \item Correcting assignments and helping students in lab sessions in the following courses:
    \begin{itemize}
        \item \course{02138}{Digital Electronics 1}
        \item \course{02139}{Digital Electronics 2}
        \item \course{02155}{Computer Architecture}
    \end{itemize}
    \item Guiding students through a project concerning error correction using Hamming code in the course \course{31011}{Engineering Practices}
\end{itemize} 

\divider

\cvevent{Research Assistant}{DTU Compute}{Sept 2020 -- Dec 2020}{Copenhagen, Denmark}
\begin{itemize}
\item Implement a priority queue in hardware as a use case to evaluate the \link{https://github.com/chiselverify/chiselverify}{ChiselVerify} hardware verification framework for the hardware construction language \link{https://www.chisel-lang.org/}{Chisel} [\cite{chiselverify}]
\end{itemize}

\divider

\cvevent{Internship}{Rheinland Air Service GmbH}{Jan 2018 --  Mar 2018}{Mönchengladbach, Germany}
\begin{itemize}
    \item Gained insight into all aspects of the maintenance process of aircrafts of the \link{https://en.wikipedia.org/wiki/ATR_(aircraft_manufacturer)}{ATR family} during an 8 week internship
\end{itemize}





% \divider

% \cvevent{Product Engineer}{Google}{23 June 1999 -- 2001}{Palo Alto, CA}

% \begin{itemize}
% \item Joined the company as employe \#20 and female employee \#1
% \item Developed targeted advertisement in order to use user's search queries and show them related ads
% \end{itemize}


% use ONLY \newpage if you want to force a page break for
% ONLY the currentc column
\newpage

\cvsection{Publications}

\nocite{*}

% \printbibliography[heading=pubtype,title={\printinfo{\faBook}{Books}},type=book]

%\divider

\printbibliography[heading=pubtype,title={\printinfo{\faFile*[regular]}{Journal Articles}}, type=article]

%\divider

%\printbibliography[heading=pubtype,title={\printinfo{\faUsers}{Conference Proceedings}},type=inproceedings]

%% Switch to the right column. This will now automatically move to the second
%% page if the content is too long.
\switchcolumn

\cvsection{Strengths}

\cvtag{Eager to learn}
\cvtag{Enduring}
\cvtag{Responsible}

\divider\smallskip

\cvtag{Computer Architecture}
\cvtag{Digital Design}
\cvtag{Verification}
\cvtag{Embedded Systems}

\cvsection{Skills}

\cvskillLvl{German}{mother tongue} %% supports X.5 values.

\cvskillLvl{English}{\link{https://en.wikipedia.org/wiki/Common_European_Framework_of_Reference_for_Languages}{C1}}

\cvskillLvl{Danish}{\link{https://en.wikipedia.org/wiki/Common_European_Framework_of_Reference_for_Languages}{C1}}

\cvskillLvl{French}{\link{https://en.wikipedia.org/wiki/Common_European_Framework_of_Reference_for_Languages}{B2}}


\divider

\cvskill{C}{5}

\cvskill{Scala}{4}

\cvskill{Java}{4}

\cvskill{C++}{3}

\cvskill{F\#}{3}

\cvskill{Python}{3}

\divider

\cvskill{Chisel}{5}

\cvskill{VHDL}{4}

\cvskill{Verilog}{3}


\divider

\cvskill{Linux}{4}

\cvskill{Windows}{5}



% \divider




\cvsection{Education}

\cvevent{B.S.\ in Electrical Engineering}{Technical University of Denmark}{Sept 2019 -- Ongoing}{}

\newpage

\cvsection{Referees}

% \cvref{name}{email}{mailing address}
\cvref{Assoc. Prof.\ Martin Schoeberl}{DTU Compute}{masca@dtu.dk}
{Richard Petersens Plads, 322\\2800 Lyngby}

\divider

\cvref{Assoc. Prof.\ Flemming Stassen}{DTU Compute}{flst@dtu.dk}
{Richard Petersens Plads, 322\\2800 Lyngby}

\end{paracol}

\end{document}
